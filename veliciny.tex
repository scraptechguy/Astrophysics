
\chapter**{Veličiny}

7 základních jednotek SI:
s, m, kg, A, K, mol, cd
(čes. sekunda, metr, kilogram, ampér, kelvin, mol, kandela)

abecední pořadí

\begingroup
\vskip\baselineskip

\def\s#1#2#3\\{%
  \noindent\hbox to 2cm{#1\hfil}%
  \noindent\hbox to 3.5cm{#2\hfil}%
  \hangindent5cm\hangafter1%
  #3
  \par
}

\s{$a$}{m} velká poloosa\\
\s{$a$}{1} expanzní parametr\\
\s{$\vec a$}{${\rm m}\,{\rm s}^{-2}$} zrychlení\\
\s{$A$}{rad} azimut\\
\s{$b$}{m} malá poloosa\\
\s{$\vec B$}{${\rm T} = {\rm N}\,{\rm A}^{\!-1}\,{\rm m}^{-1}\!$ [tesla]} magnetické pole\\
\s{$c$}{m/s} rychlost světla ve vakuu\\
\s{$C$}{F = C/V [farad]} kapacita\\
\s{$e$}{1} číselná excentricita\\
%\s{$e$}{C} elementární náboj\\
\s{$E$}{J [\kern.5pt joule]} energie\\
\s{$\vec E$}{N/C = V/m} elektrické pole\\
\s{$f$}{${\rm Hz} = 1/{\rm s}$ [Hertz]} frekvence\\
\s{$\vec F$}{N [newton]} síla\\
\s{$g$}{${\rm m}\,{\rm s}^{-2}$} gravitační zrychlení\\
\s{$G$}{${\rm kg}^{-1}\,{\rm m}^3\,{\rm s}^{-2}$} gravitační konstanta\\
\s{$h$}{J\,s} Planckova konstanta\\
\s{$h$}{rad} výška nad obzorem\\
\s{$H$}{1/s} Hubbleův parametr\\
\s{$I$}{A = C/s [ampér]} elektrický proud\\
\s{$I$}{${\rm kg}\,{\rm m}^2$} moment setrvačnosti\\
\s{$J$}{W/sr} zářivost\\
\s{$J'$}{cd [kandela]} svítivost\\
\s{$k$}{cyklů/m} vlnové číslo\\
\s{$k_{\rm B}$}{J/K} Boltzmannova konstanta\\
\s{$k$}{N/m} tuhost pružiny\\
\s{$l$}{m} délka\\
\s{$L$}{${\rm H} = {\rm T}\,{\rm m}^2\,{\rm A}^{\!-1}\!$ [henry]} indukčnost\\
\s{$\vec L$}{${\rm kg}\,{\rm m}^2\,{\rm s}^{-1}$} moment hybnosti\\
\s{$m$}{kg} hmotnost\\
\s{$M_{\rm mol}$}{mol} molární hmotnost\\
\s{$\vec M$}{${\rm N}\,{\rm m}$} moment síly\\
\s{$n$}{${\rm m}^{-3}$} koncentrace\\
\s{$N$}{1} počet částic\\
\s{$N_{\rm mol}$}{mol} látkové množství\\
\s{$p$}{${\rm Pa} = {\rm N}\,{\rm m}^{-2}$ [pascal]} tlak\\
\s{$\vec p$}{${\rm kg}\,{\rm m}\,{\rm s}^{-1}$} hybnost\\
\s{$P$}{s} perioda\\
\s{$P$}{W = J/s [watt]} výkon\\
\s{$q$}{C [coulomb]} náboj\\
\s{$Q$}{J} teplo\\
\s{$\vec r$}{m} polohový vektor\\
\s{$\hat r$}{1} jednotkový vektor\\
\s{$R$}{$\Omega = {\rm V}/{\rm A}$ [ohm]} elektrický odpor\\
\s{$s$}{m} dráha\\
\s{$S$}{${\rm m}^2$} plocha\\
\s{$t$}{s} čas\\
\s{$t$}{rad} hodinový úhel\\
\s{$T$}{K [kelvin]} teplota\\
\s{$U$}{V = J/C [volt]} napětí\\
\s{$v_{\rm r}$}{m/s} radiální rychlost\\
\s{$\vec v$}{m/s} rychlost\\
\s{$V$}{${\rm m}^3$} objem\\
\s{$W$}{J} práce\\

\s{}{} \\

\s{$\alpha$ [alfa]}{rad} rektascenze\\
\s{$\delta$ [delta]}{rad} deklinace\\
\s{$\Delta$}{m} dráhový rozdíl\\
\s{$\varepsilon$ [epsilon]}{${\rm F}/{\rm m} = {\rm N}^{-1}\,{\rm C}^2\,{\rm m}^{-2}$} permitivita (čes. nepronikavost)\\
\s{$\theta$ [théta]}{rad} hvězdný čas\\
\s{$\kappa$ [kapa]}{${\rm m}^2\,{\rm kg}^{-1}$} opacita (neprůhlednost)\\
\s{$\lambda$ [lambda]}{m} vlnová délka\\
\s{$\mu$ [mí]}{${\rm H}/{\rm m} = {\rm N}\,{\rm A}^{\!-2}$} permeabilita (propustnost)\\
\s{$\varphi$ [fí]}{rad} úhel\\
\s{$\Phi$}{${\rm W}\,{\rm m}^{-2}$} zářivý tok\\
\s{$\Phi_{\rm m}$}{${\rm T}\,{\rm m}^2$} magnetický tok\\
\s{$\varrho$ [ró]}{${\rm kg}\,{\rm m}^{-3}$} hustota\\
\s{$\sigma$ [sigma]}{${\rm W}\,{\rm m}^{-2}\,{\rm K}^{-4}$} Stefanova--Boltzmannova konstanta\\
\s{$\omega$ [omega]}{rad/s} úhlová rychlost\\
\s{$\Omega$}{sr [steradián]} prostorový úhel\\

\s{}{} \\

%alternativní dvoupísmenné značení?

%\s{AZ}{rad} azimut\\
%\s{DE}{rad} deklinace\\
%\s{HA}{rad} hodinový úhel\\
%\s{HT}{rad} výška nad obzorem\\
%\s{RA}{rad} rektascenze\\
%\s{ST}{rad} hvězdný čas\\

\s{}{} \\

\setbox0=\hbox{\verbtt|._.|}\s{\box0}{} \\



%\s{$q$}{m} pericentrum\\
%\s{$Q$}{m} apocentrum\\


\tabcaption{Tabulka kinematických veličin odvozených od m, s.}
\table{
\offinterlineskip
\halign{
\strut#\hfil & #\hfil & #\hfil & #\hfil & #\hfil & #\hfil & #\hfil & #\hfil \cr
               & ${\rm m}^{-3}$ & ${\rm m}^{-2}$ & ${\rm m}^{-1}$ & 1   & m        & ${\rm m}^2$ & ${\rm m}^3$ \cr
${\rm s}^{-2}$ &                &                &                &     & $\vec a$ &             &             \cr
${\rm s}^{-1}$ &                &                &                &     & $\vec v$ &             &             \cr
1              &                &                &                &     & $\vec r$ &             &             \cr
s              &                &                &                & $t$ &          &             &             \cr
${\rm s}^2$    &                &                &                &     &          &             &             \cr
}
}


\tabcaption{Tabulka dynamických veličin odvozených od kg, m, s.}
\table{
\offinterlineskip
\halign{
\strut#\hfil & #\hfil & #\hfil & #\hfil & #\hfil & #\hfil & #\hfil & #\hfil \cr
               & kg             & kg             & kg             & kg  & kg       & kg          & kg          \cr
               & ${\rm m}^{-3}$ & ${\rm m}^{-2}$ & ${\rm m}^{-1}$ & 1   & m        & ${\rm m}^2$ & ${\rm m}^3$ \cr
${\rm s}^{-2}$ &                &                &                &     &          &             &             \cr
${\rm s}^{-1}$ &                &                &                &     &          &             &             \cr
1              &                &                &                & $m$ &          &             &             \cr
s              &                &                &                &     &          &             &             \cr
${\rm s}^2$    &                &                &                &     &          &             &             \cr
}
}


% kg, m , s, C

pro elektrické veličiny není jednoduché udělat tabulku, protože:

$[q] = $

$[I] = {\rm C}\,{\rm s}^{-1}$

$[R] = {\rm kg}\,{\rm m}^2\,{\rm s}^{-3}\,{\rm A}^{-2}$

$[U] = {\rm kg}\,{\rm m}^2\,{\rm s}^{-3}\,{\rm A}^{-1}$

$[\vec E] = {\rm kg}\,{\rm m}\,{\rm s}^{-3}\,{\rm A}^{-1}$

$[\vec B] = {\rm kg}\,{\rm s}^{-2}\,{\rm A}^{-1}$

$[\mu] = {\rm A}\,{\rm m}^{-2}$!!<- check

$[\varepsilon] = $

$[L] = $

$[C] = $

% B = mu I/(2 pi r) ... propustnost
% E = 1/(4 pi eps) q/r^2 ... nepronikavost
