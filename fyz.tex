% fyz.tex
% Astrofyzika pro gymnázia
% Miroslav Broz (miroslav.broz@email.cz), Rostislav Broz
% vers. Jan 21st 2022

\input config
\input povetron
\input kniha
\input title

\tableofcontents

\input uvod
\input veliciny

%%%%%%%%%%%%%%%%%%%%%%%%%%%%%%%%%%%%%%%%%%%%%%%%%%%%%%%%%%%%%%

\vfill\eject

\begingroup
\let\oldpar=\par\def\par{\oldpar\leavevmode}
\obeylines\obeyspaces

řazení dle vzdálenosti:

Orientace na obloze
Meteory
Meteority
Země
Měsíc
Planety
Planetky a komety
Slunce
Cizí hvězdy
Cizí planety
Akreční disky
Mezihvězdný plyn a prach
Černé díry
Zakřivení času a prostoru
Galaxie
Cizí galaxie
Mezigalaktický plyn
Temná látka
Milimetrové pozadí
Expanze prostoru
Temná energie


řazení dle jiných učebnic:

Optika dalekohledů
Mechanika planet
Termika hvězd
Oscilace hvězd
Elektřina a magnetismus Země
Částice a záření v mezihvězdném prostoru
Expanze vesmíru


zase jiné řazení:

Země a padající jablko
Země a skleníkový jev
Měsíc a ragbyový míč
Slunce a varná konvice
Mars a trojný bod
Jupiter a tři tělesa
Neptun a kyvadlo
Asteroidy a exploze
Meteority a radioaktivita
Hvězdy a fotoelektrický jev
Hvězdy a duha
Mezihvězdné prostředí a mlha
Akreční disky a ???
Černé díry a ???
Magnetické pole a polární záře


\endgroup

\hrule


různé rovnice a pojmy:

dalekohled zmenšuje předmět, na jehož obraz se díváme zblízka

zrcadlo nepřevrací stranově, nýbrž hloubkově

radiometrie

kružnice
elipsa
parabola
hyperbola

potenciální energie

kinetická energie

gravitační zákon

stoupání tlaku

práce konaná silou

paralaxa

1. Keplerův zákon

$$r(\varphi) = {p\over 1+e\cos\varphi}$$

2. Keplerův zákon

$$\vec L = \vec r\times\vec v = {\rm const.}$$

3. Keplerův zákon

$${a^3\over T^2} = {G(m_1+m_2)\over 4\pi^2}$$

opacita

fázový diagram

trojný bod

stavová rovnice plynu

stavová rovnice pevné látky

Fridmanova rovnice

expanzní parametr (bezrozměrný)
$a(t)$

$H \doteq 70\,{\rm km}\,{\rm s}^{-1}\,{\rm Mpc}$
$d = 1\,{\rm Mpc}$,
galaxie M\,33 v souhvězdí Trojúhelníku
i když tato je v místní skupině
$v_{\rm r} = 70\,{\rm km}/{\rm s}$


$H \doteq 2{,}3\cdot10^{-18}\,{\rm s}^{-1}$
přibližné stáří vesmíru
$1/H \doteq 4{,}4\cdot10^{17}\,{\rm s} \doteq 14$~miliard roků

rychlost vln

$$c = \lambda f$$
je základní vlastnost záření frekvence nebo vlnová délka?
na 1 místě kmitám nahoru--dolu s frekvencí~$f$,
odtud se šíří vlny rychlostí~$c$,
odkud plyne jejich vlnová délka~$\lambda$,
čili~$f$

energie fotonu

$$E = hf$$

Planckova funkce
intezita

$$I_f = {2hf^3\over c^2}{1\over\exp\left({hf\over kT}\right)-1}$$

Stefanův--Boltzmannův zákon
do poloprostoru

$$\Phi = \sigma T^4$$



%%%%%%%%%%%%%%%%%%%%%%%%%%%%%%%%%%%%%%%%%%%%%%%%%%%%%%%%%%%%%%

\input literatura

\bye

